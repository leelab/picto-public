\begin{DoxyAuthor}{Author}
Joey Schnurr, Mark Hammond, Matt Gay 
\end{DoxyAuthor}
\begin{DoxyVersion}{Version}
2.\-0.\-1 
\end{DoxyVersion}
\begin{DoxyDate}{Date}
Last modified\-: January 24, 2014
\end{DoxyDate}
This is the Developer's Guide for the \hyperlink{namespace_picto}{Picto} suite of experimental control applications and associated Picto\-Box hardware. If you are an investigator utilizing the software as part of your research then you probably want the Picto\-Manual.\-docx located in \hyperlink{namespace_picto}{Picto}. If you are modifying or attempting to build a Pictobox, you want the Pictobox\-Documentation.\-docx from \hyperlink{namespace_picto}{Picto}. If you are interested in learning how the system itself works and/or modifying the system, then this guide is the right place to start.\hypertarget{index_devdocs_getting_started_introduction}{}\section{Introduction}\label{index_devdocs_getting_started_introduction}
The \hyperlink{namespace_picto}{Picto} suite of experimental control applications is utilized by researchers to perform neurobiological experiments in the study of decision making. The system consists of serveral components\-:
\begin{DoxyItemize}
\item The \hyperlink{class_director}{Director} component runs the experimental \char`\"{}game\char`\"{} that the test subject \char`\"{}plays.\char`\"{} This component takes in user input from a mouse input, eye tracker, joystick, or any other device that can provide analog x/y data. Inputs are monitored to control the flow of the \char`\"{}game\char`\"{} and all user input, reward, and experimental state data are sent over the network to the \hyperlink{namespace_picto}{Picto} \hyperlink{class_server}{Server} Component. A hardware line connects the \hyperlink{class_director}{Director} to the Proxy component and an alignment code is sent over this line periodically for the purpose of time alignment.
\item The Proxy component (\hyperlink{class_proxy_main_window}{Proxy\-Main\-Window}) runs on a neural data acquisition computer (currently supported models are Plexon and T\-D\-T). It collects data from the acquisition system, including neural spikes, local field potential data and alignment codes coming in from the \hyperlink{class_director}{Director}, repackages the data and sends it over the network to the \hyperlink{namespace_picto}{Picto} \hyperlink{class_server}{Server} component.
\item The \hyperlink{namespace_picto}{Picto} \hyperlink{class_server}{Server} component is the main hub of the application suite. This component takes in data from the Proxy and \hyperlink{class_director}{Director}, aligns the timestreams from the two components based on the alignment codes and saves the data to disk in a Session file. It also makes Director/\-Proxy data available in real time for monitoring from the Workstation component. Lastly, the \hyperlink{namespace_picto}{Picto} \hyperlink{class_server}{Server} acts as the glue that binds the Workstation to the Proxy and \hyperlink{class_director}{Director}. Whenever the Workstation sends a command to be handled on either the Proxy or the \hyperlink{class_director}{Director}, the command is sent to the \hyperlink{class_server}{Server}. The \hyperlink{class_server}{Server} takes care of forwarding the command to the appropriate component and informs the Workstation when/if the component complies with the command directive.
\item The Workstation component is the Researcher's window into the \hyperlink{namespace_picto}{Picto} system. It includes a \hyperlink{class_remote_viewer}{Remote\-Viewer} which Researchers use to start/stop Experimental sessions, control them and monitor their activity. It includes a \hyperlink{class_state_edit_viewer}{State\-Edit\-Viewer}, used to design the Experiment that runs on the \hyperlink{class_director}{Director} and the Analysis that is used to extract data from session files. It includes a \hyperlink{class_test_viewer}{Test\-Viewer} for locally testing and debugging Experimental designs and Analyses. Lastly, the Workstation includes a \hyperlink{class_replay_viewer}{Replay\-Viewer} which is used to playback Session files, record Session activity to a video file, and run Session Analyses.
\end{DoxyItemize}

While Neurobiological research is the main focus on the application suite, we have made every effort to design it in such a way that it will be generalizable and extendable for other types of applications as well. In this vain, and for the purpose of training users, the suite can be used without an attached Proxy component. In its essence, \hyperlink{namespace_picto}{Picto} is a system for designing and running experiments consisting of precisely timed, carefully controlled stimuli and user feedback with an optional capability for tracking neurobiological data and aligning it with the behavioral stream.\hypertarget{index_devdocs_getting_started_prerequisites}{}\section{Prerequisites}\label{index_devdocs_getting_started_prerequisites}
This document assumes that the reader has experience with C++ based software development and is familiar with the Qt framework. Many many details and tutorials about getting started with C++ and Qt are available online. We also assume that the reader has a basic level of comfort with the command-\/line interface, as it will be used to prepare a build environment.\hypertarget{index_devdocs_getting_started}{}\section{Getting Started}\label{index_devdocs_getting_started}
In order to get up to speed with \hyperlink{namespace_picto}{Picto} development, you will need to create a development environment, become familiar with the various files used in the \hyperlink{namespace_picto}{Picto} system and familiarize yourself with the code base. The following pages should help guide you through these tasks.
\begin{DoxyItemize}
\item \hyperlink{build_environment_preparation}{Create a Picto Development Environment} -\/ Use this page to set up your development environment to build and edit \hyperlink{namespace_picto}{Picto} code.
\item \hyperlink{user_files}{Understand Picto User Files} -\/ Use this page to gain a broad understanding of the files used generated and used by \hyperlink{namespace_picto}{Picto}.
\item \hyperlink{first_code_look}{First Look At Code} -\/ Use this as a guide to becoming familiar with the \hyperlink{namespace_picto}{Picto} code base.
\end{DoxyItemize}\hypertarget{index_to_do}{}\section{To Do...}\label{index_to_do}

\begin{DoxyItemize}
\item \hyperlink{known_bugs}{Known Bugs} -\/ This page contains the list of current known bugs.
\item \hyperlink{future_directions_software}{Future Directions in Software} -\/ This page contains some ideas for features that we would like to add to the \hyperlink{namespace_picto}{Picto} software. 
\end{DoxyItemize}